%%%%%%%%%%%%%%%%%%%%%%%%%%%%%%%%%%%%%%%%%
% Journal Article
% LaTeX Template
% Version 2.0 (February 7, 2023)
%
% This template originates from:
% https://www.LaTeXTemplates.com
%
% Author:
% Vel (vel@latextemplates.com)
%
% License:
% CC BY-NC-SA 4.0 (https://creativecommons.org/licenses/by-nc-sa/4.0/)
%
% NOTE: The bibliography needs to be compiled using the biber engine.
%
%%%%%%%%%%%%%%%%%%%%%%%%%%%%%%%%%%%%%%%%%

%----------------------------------------------------------------------------------------
%	PACKAGES AND OTHER DOCUMENT CONFIGURATIONS
%----------------------------------------------------------------------------------------

\documentclass[
	a4paper, % Paper size, use either a4paper or letterpaper
	10pt, % Default font size, can also use 11pt or 12pt, although this is not recommended
	% unnumberedsections, % Comment to enable section numbering
	% twoside, % Two side traditional mode where headers and footers change between odd and even pages, comment this option to make them fixed
]{LTJournalArticle}

\usepackage[ngerman]{babel}
\usepackage{pdfpages}
\usepackage{listings}

\usepackage{lipsum}
% \newcommand{\lipsum}{TODO}

\usepackage{etoolbox}
\makeatletter
\patchcmd{\chapter}{\if@openright\cleardoublepage\else\clearpage\fi}{}{}{}
\makeatother

\addbibresource{zotero.bib} % BibLaTeX bibliography file
\addbibresource{mlcore.bib} % BibLaTeX bibliography file

\runninghead{Predictive Analytics-as-a-Service} % A shortened article title to appear in the running head, leave this command empty for no running head

\footertext{\textit{Fernuniversität Hagen} (2026)} % Text to appear in the footer, leave this command empty for no footer text

\setcounter{page}{1} % The page number of the first page, set this to a higher number if the article is to be part of an issue or larger work

%----------------------------------------------------------------------------------------
%	TITLE SECTION
%----------------------------------------------------------------------------------------

\title{Predictive Analytics-as-a-Service\\\normalsize{Fachpraktikum Softwareentwicklung mit Methoden der Künstlichen Intelligenz}} % Article title, use manual lines breaks (\\) to beautify the layout

% Authors are listed in a comma-separated list with superscript numbers indicating affiliations
% \thanks{} is used for any text that should be placed in a footnote on the first page, such as the corresponding author's email, journal acceptance dates, a copyright/license notice, keywords, etc
\author{%
	Nikolaos Psycharis\\ 
	Hamza Gueni\\ 
	Joshua Allius\\
	\textbf{Erstellt:} \today\space
	\thanks{
		Kontakt: \\ 
		\href{mailto:nikolaos.psycharis@studium.fernuni-hagen.de}{nikolaos.psycharis@studium.fernuni-hagen.de}\\ 
		\href{mailto:hamza.gueni@studium.fernuni-hagen.de}{hamza.gueni@studium.fernuni-hagen.de}\\ 
		\href{mailto:joshua.allius@fernuni-hagen.de}{joshua.allius@fernuni-hagen.de}
	}
}

% Affiliations are output in the \date{} command
\date{\footnotesize Fernuniversität Hagen}

% Full-width abstract
\renewcommand{\maketitlehookd}{%
	\begin{abstract}
		Im Rahmen des Fachpraktikums \glqq Softwareentwicklung mit Methoden der Künstlichen Intelligenz\grqq\space wurde das Projekt \glqq Predictive Analytics-as-a-Service\grqq\space (PAaaS) erstellt. Dabei handelt es sich um eine skalierbare Web-Anwendung, um die Anwendung verschiedener KI-Algorithmen zur Analyse diverser Datensätze zu vereinfachen. In einer containerisierten Microservice-Architektur wird unter anderem FastAPI, React und Celery / Redis verwendet, um das \glqq MLCore\grqq\space genannte Backend anzusteuern. MLCore bietet dabei viele Möglichkeiten für predictive analysis und reduction analysis. Neben erweiterten Analyseoptionen liegen mögliche Verbesserungen im Bereich der Einstiegshilfe, der internen Datenspeicherung und der allgemeinen Nutzungsleichtigkeit vor.
	\end{abstract}
}

%----------------------------------------------------------------------------------------

\begin{document}

\maketitle % Output the title section

%----------------------------------------------------------------------------------------
%	ARTICLE CONTENTS
%----------------------------------------------------------------------------------------

\clearpage
\section{Übersicht}
\label{sec:uebersicht}
Aktuelle Methoden für prädiktive Analyseverfahren (\glqq Predictive Analytics\grqq) sind mitunter schwierig zu bedienen oder stehen lediglich als Bibliothek zur Verfügung und sind so nicht ohne weiteres in verschiedene Systeme einbindbar. Eine einheitliche Schnittstelle oder (grafische) Bedienoberfläche für solche Verfahren existiert derzeit nicht.


\subsection{Motivation \& Zielsetzung}
\label{subsec:Motivation}
Das Ziel hinter \code{PAaaS} ist die Schaffung eines Nutzerinterfaces sowie einer geeigneten RESTful API für verschiedene Methoden der prädiktiven Analyse. Dabei sollte das Tool leicht bedienbar, modular und skalierbar sein. 

Zuerst wurde das Backend sowie die API spezifiziert, mit einem Frontend was anschließend auf dieser API aufbaut. Ebenso wichtig war bei der Entwicklung ein Fokus auf allgemein gebräuchliche Tools, um die spätere Weiterentwicklung, Wartung und das Deployment zu vereinfachen.

\subsection{Aufbau der Arbeit}
Im Folgenden wird in \refsec{sec:struktur}-\ref{sec:webui} die Methodik, Organisation und Projektumsetzung erläutert: zunächst wird kurz die Idee von \code{PAaaS} und der Aufbau betrachtet (siehe \refsec{sec:struktur}). Dabei wird auf die einzelnen Komponenten und deren Zusammenspiel in Form von Containerisierung eingegangen. Auch wird in \refsec{sec:workflow} der Workflow während der Projekt-Entwicklung dargestellt. Anschließend wird detailliert auf den Aufbau der Datenbank und die Strukturierung der Daten eingegangen (siehe \refsec{sec:db}). Danach folgt eine detaillierte Darstellung von MLCore, also dem Modell-Training und der eigentlichen \glqq Predictive Analysis\grqq in \refsec{sec:mlcore}. Außerdem wird die Funktionsweise der RESTful API und der Web-UI dargestellt (siehe \refsec{sec:webui}) 

Anschließend folgt in \refsec{sec:eval} eine Evaluation der Projektergebnisse gegenüber der in \refsubsec{subsec:Motivation} genannten Motivation und Zielsetzung.

Abschließend wird in \refsec{sec:perspektive} auf einige Erweiterungen und Ausbaumöglichkeiten eingegangen. Diese stellen interessante Optionen für eine Weiterentwicklung dar, welche jedoch im Zeitrahmen des Projektes nicht mehr umgesetzt werden konnten. 

% \subsection{Anmerkung der Autoren}
% \label{subsec:anmerkung}
% Dieser Bericht sowie der \code{PAaaS}-Code wurde im Rahmen eines Fachpraktikums an der Fernuniversität Hagen umgesetzt. Dabei wurden innerhalb des Projektes die Aufgaben aufgeteilt.

% Die Komponenten Frontend sowie MLCore wurden durch Nikolaos Psycharis implementiert. Das Projektdesign, die Datenbank sowie Anbindung dieser an FastAPI und MLCore wurden durch Hamza Gueni umgesetzt. Die Containerisierung/Skalierung und Endpoints wurden durch Joshua Allius umgesetzt.


\section{Struktur}
\label{sec:struktur}
\ja
Bei dem Design des Projektes wurde ein großer Fokus auf Modularisierung und Skalierbarkeit gelegt. Ebenso wurden primär weit verbreitete Tools eingesetzt, um die Weiterentwicklung und Wartung des Projektes zu vereinfachen. Folglich sind die einzelnen Komponenten über bestimmte Schnittstellen miteinander verbunden. Innerhalb dieser ist eine Modifizierung oder ein Austausch der Komponenten unabhängig der anderen Komponenten umzusetzen. Dabei wurde Docker als Containerisierungslösung eingesetzt, da Docker eine sehr weit verbreitete Lösung darstellt \cite{sollfrankEvaluatingDockerLightweight2021}.

\subsection{Komponenten}
\label{subsec:komponenten}
\ja
Die einzelnen Komponenten des Projekts werden als containerisierte Microservices ausgerollt. Dabei werden die folgenden Komponenten verwendet:

\begin{enumerate}
    \item Frontend Node: eine React-basierte Weboberfläche zur vereinfachten Nutzung durch den Anwender
    \item API Node: eine FastAPI-basierte API, über welche sämtliche Funktionen des Projektes angestoßen werden können
    \item Worker Node: ein Python-Container, in dem die MLCore-bezogenen Tasks ausgeführt werden
    \item MySQL: eine SQL-basierte Datenbank
    \item Redis: Messaging Host für Zuweisung von Celery Tasks von / an verschiedene API / Worker Nodes
    \item Flower: eine Weboberfläche zum Redis-Monitoring
\end{enumerate}

Besonders sind dabei die Frontend, API und Worker Nodes zu betrachten. Die verbleibenden Komponenten sind unmodifizierte 3rd-party Softwares.

% TODO:
\todo{Aufbau von frontend, API, Worker. Warum wurden diese Tools gewählt?}

MySQL wurde aufgrund der weiten Verbreitung des Tools verwendet. So war MySQL 2023 die zweit-meistverwendeste Datenbank nach Oracle \cite{DBEnginesRanking}. Oracle wurde allerdings aufgrund der Lizenzbedingungen für dieses Projekt nicht in Betracht gezogen. Zudem wurde speziell eine SQL-basierte Datenbank verwendet, um die Vorteile einer effizienten Speicherung für strukturierte Metadaten zu nutzen. Dabei ist zu beachten, dass in der aktuellen Version Nutzer-Uploads nur eingeschränkt in der Datenbank gespeichert werden (siehe Sektion \ref{sec:perspektive}). Für eine direkte Speicherung z.B. von .csv-Uploads wären auch dokumentbasierte Datenbanken wie beispielsweise MongoDB in Betracht zu ziehen gewesen \cite{DocumentStoresDBEngines}.

Redis und Celery sind verbreitete Tools für Messaging. Diese Tools wurden aufgrund bestehender Erfahrungen im Team sowie der leichten Implementierung in Python gewählt. Flower wurde folglich als Monitoring-Tool für Redis gewählt.

\begin{figure*}
    \centering
    \includegraphics[width=\textwidth]{../structure_release.png}
    \caption{Containerisierte Projekt-Struktur}
    \label{fig:structure}
\end{figure*}

\subsection{Containerisierung}
\label{subsec:container}
\ja
\lipsum

\subsection{Datenbank}
\label{subsec:db}
\hg
\lipsum

\section{Workflow}
\ja
\label{sec:workflow}
% TODO: pytest, CI
\lipsum
\section{Datenbank}
\label{sec:db}
\code{PAaaS} folgt einer Kette. Eine Nutzerin oder ein Nutzer legt ein Dataset an. Ein Upload wird als Dataset-Version gespeichert. Darauf wird ein Machine-Learning-Problem definiert. Aus dem Training entstehen Modelle, und aus Prediction-Runs entstehen Ausgaben.
Solange das Projekt klein ist, bleibt diese Kette übersichtlich. Sobald mehrere Datasets, Modelle und Jobs parallel existieren, wird die Nachvollziehbarkeit wichtiger als die einzel-
nen Schritte. Dann geht es um Fragen wie: Welche Dataset-Version hat ein Modell trainiert. Welches Modell hat eine Vorhersage erzeugt. Wer hat einen Job angestoßen. Die Datenbank hält diese Beziehungen explizit fest. Jedes Objekt wird einmal gespeichert und über stabile Identifier mit dem vorherigen Schritt verbunden. Abbildung \ref{fig:db1} zeigt die fachliche Lineage. Abbildung \ref{fig:db2} zeigt, wie diese Lineage als Tabellen und Beziehungen umgesetzt ist.

\subsection{Grundlagen und Vorgehen}
Dieser Abschnitt führt die Begriffe ein, die später im Text vorkommen, und beschreibt das Vorgehen im Projekt.
Eine Datenbank speichert Daten so, dass sie nach dem Ende eines Programms weiterhin verfügbar sind. Eine relationale Datenbank organisiert Daten in Tabellen. Tabellen haben Zeilen und typisierte Spalten. Die Menge aus Tabellen, Spalten und Regeln nennt
man Schema. MySQL ist ein relationales Datenbankmanagementsystem. Es speichert und fragt Daten mit SQL ab und kann Regeln wie Schlüsselbeziehungen direkt im Schema erzwingen \cite{MySQL80Reference}.

Nicht-relationale Datenbanken speichern Daten in anderen Strukturen, zum Beispiel als Dokumente, Key-Value-Paare oder Graphen. In \code{PAaaS} sind die Kernobjekte und ihre Beziehungen stabil. Datasets, Versionen, Probleme, Modelle, Jobs und Predictions haben
eine klare Struktur. Dafür passt ein relationales Schema gut. 

Ein Primärschlüssel identifiziert eine Zeile eindeutig. Ein Fremdschlüssel speichert den Primärschlüsselwert einer anderen Tabelle und drückt damit eine Beziehung aus. Fremdschlüssel schützen die Integrität, weil sie Verweise auf nicht existierende Zeilen verhindern \cite{MySQL80Reference}. Im Schema werden UUIDs als Identifier verwendet. Eine UUID ist ein standardisiertes 128-Bit-Format, das für unabhängige Generierung gedacht ist \cite{leachUniversallyUniqueIDentifier2005}. Im Projekt werden UUIDs als \code{CHAR(36)} gespeichert. Das macht sie beim Debugging lesbar, auch in Logs und in SQL-Abfragen.

Ein Teil der Informationen in \code{PAaaS} hat keine feste Struktur. Dataset-Profile hängen von den Spalten eines Uploads ab. Modellmetriken hängen vom Algorithmus und der Aufgabe ab. JSON ist ein standardisiertes Format für strukturierte Daten mit Objekten
und Arrays \cite{brayJavaScriptObjectNotation2017}. MySQL bietet dafür einen nativen JSON-Datentyp. Damit lassen sich solche Metadaten speichern, ohne ständig neue optionale Spalten anzulegen \cite{135JSONData}.

\code{PAaaS} speichert außerdem Verweise auf große Artefakte, etwa CSV-Dateien, Modell-Binaries und Prediction-Outputs. Deren Speicherorte werden als URI abgelegt. Eine URI ist ein standardisierter Bezeichner für Ressourcen mit klarer Syntax \cite{berners-leeUniformResourceIdentifier2005}. So bleibt die Datenbank auf strukturierte Daten und Metadaten fokussiert, während Dateien in der Storage-Schicht liegen.

Im Projekt wurde ein Lineage-First-Ansatz gewählt. Jeder Schritt im Workflow wird als eigene Tabellenzeile gespeichert. Jedes spätere Objekt speichert den Identifier des Schrittes, von dem es abhängt. Damit wird Nachvollziehbarkeit zu einer Query-Aufgabe.
Für einzelne Updates spielt zusätzlich Konsistenz über mehrere Zeilen eine Rolle. Transaktionen bündeln mehrere SQL-Statements zu einer Einheit. Isolation Levels beschreiben, wie parallele Transaktionen miteinander interagieren \cite{1772InnoDBTransaction}.

\subsection{Schema und Implementierung}
Dieser Abschnitt verbindet den Workflow mit dem konkreten Schema und dem Python-Zugriff.

Abbildung \ref{fig:db1} zeigt die fachliche Kette. Das Schema bildet diese Kette mit Fremdschlüsseln nach. Eine Dataset-Version zeigt auf ihr Dataset. Ein Problem zeigt auf eine Dataset-Version. Ein Modell zeigt auf ein Problem. Jobs und Predictions zeigen auf das jeweilige Modell. Jobs speichern zusätzlich, wer die Aktion angefordert hat.

\begin{figure}[!h]
    \centering
    \includegraphics[width=\linewidth]{./figures/db1.png}
    \caption{Konzeptionelle Lineage von user-owned Datasets bis zu Predictions.}
    \label{fig:db1}
\end{figure}

Abbildung \ref{fig:db2} folgt derselben Reihenfolge wie der Workflow. Dadurch wird auch klar, warum Joins für Detailansichten naheliegend sind. Wenn die UI zu einem Modell zusätzlich den Dataset-Namen anzeigen soll, gibt es dafür einen sauberen Join-Pfad.

Der Python-Zugriff ist in db.py umgesetzt und wird von der API genutzt. Das Modul folgt dem Connection- und Cursor-Muster, wie es in der Python DB-API 2.0 beschrieben ist \cite{PEP249Python} Die meisten Funktionen folgen einem einfachen Rhythmus. Create-Funktionen schreiben eine Zeile und geben die UUID zurück. Get-Funktionen lesen eine Zeile über den Primärschlüssel. Update-Funktionen ändern nur die übergebenen Felder. Für Listenansichten sind Filter und Pagination eingebaut, damit die API nicht ganze Tabellen in den Speicher lädt.

Eine Operation ist kritischer als normale CRUD-Aufrufe. Pro ML-Problem soll es genau ein Production-Modell geben. Ein Wechsel betrifft mehrere Zeilen und läuft daher als Transaktion. Dabei wird die ML-Problem-Zeile gesperrt, ein vorheriges Production-
Modell bei Bedarf archiviert, das neue Modell als Production markiert und der Production-Pointer aktualisiert. Abbildung \ref{fig:db3} zeigt die Schritte. Transaktionen und Isolation-Regeln liefern die nötige Konsistenz bei parallelen Requests \cite{1772InnoDBTransaction}.

\begin{figure}[!h]
    \centering
    \includegraphics[height=\linewidth*2/3]{./figures/db3.png}
    \caption{Production-Model-Switch als konsistente Datenbankoperation.}
    \label{fig:db3}
\end{figure}

\begin{sidewaysfigure*}[]
    \centering
    \includegraphics[width=\textwidth]{./figures/db2_hd.png}
    \caption{Entity-Relationship-Diagramm des \code{PAaaS}-Datenbankschemas.}
    \label{fig:db2}
\end{sidewaysfigure*}

\clearpage

\subsection{Evaluation}
Die Evaluation fokussiert auf funktionale Korrektheit. Der wichtigste Nachweis ist ein lokaler Smoke Test, der einen vollständigen Happy Path ausführt. Er legt eine Nutzerin oder einen Nutzer an, erzeugt ein Dataset, eine Dataset-Version, ein ML-Problem, ein
Modell, einen Trainingsjob und eine Prediction. Danach werden Datensätze wieder gelesen und es wird geprüft, ob die Identifier passen und die Fremdschlüsselbeziehungen gültig sind. Damit ist gezeigt, dass das Schema die beabsichtigte Insert-Reihenfolge unterstützt und die Helper-Funktionen konsistent arbeiten.

Zwei SQL-Abfragen verdeutlichen das Traceability-Ziel. Abbildung \ref{lst:db1} listet Modelle zu einem
ML-Problem. Abbildung \ref{lst:db2} rekonstruiert die Lineage hinter einer Prediction über Joins ent-
lang der Workflow-Tabellen.

\begin{figure*}[!h]
    \centering
    \begin{verbatim}
SELECT id, name, algorithm, status, created_at
FROM models
WHERE problem_id = %s
ORDER BY created_at DESC;
    \end{verbatim}
    \caption{SQL: \code{models} für ein bestimmtes Problem ausgeben.}
    \label{lst:db1}
\end{figure*}

\begin{figure*}[!h]
    \centering
    \begin{verbatim}
SELECT
p.id AS prediction_id,
m.id AS model_id,
mp.id AS problem_id,
dv.id AS dataset_version_id,
d.id AS dataset_id,
d.name AS dataset_name
FROM predictions p
JOIN models m ON m.id = p.model_id
JOIN ml_problems mp ON mp.id = m.problem_id
JOIN dataset_versions dv ON dv.id = mp.dataset_version_id
JOIN datasets d ON d.id = dv.dataset_id
WHERE p.id = %s;
    \end{verbatim}
    \caption{SQL: Ein \code{prediction} zu dem zugehörigen \code{dataset} zurückverfolgen.}
    \label{lst:db2}
\end{figure*}

Diese Evaluation ist bewusst praxisnah. Sie zeigt, dass die Datenbank den Projekt-Workflow Ende zu Ende abbildet. Sie misst keine Performance unter Last und deckt keine umfangreichen Parallelitäts- oder Stress-Szenarien ab.
\section{MLCore}
\label{sec:mlcore}
MLCore ist das Subsystem für maschinelles Lernen, implementiert unter \code{src/mlcore}. Es fokussiert sich auf Data Profiling und Vorverarbeitung, Model Training, Evaluation und Explainability der Modelle sowie Vorhersage Workflows.

% ########################################################

\subsection{Designziele}
\label{subsec:mlcore:design}
Das \code{mlcore} Package enthält die Ende-zu-Ende-Logik für maschinelles Lernen, die aus den Workers ausgeführt wird. Es ist verantwortlich sowohl für das Data Profiling und die Vorverarbeitung als auch für die Modellauswahl durch vordefinierte Presets sowie für das Training, die Evaluation der Modelle, ihre Explainability und die Vorhersage.

Dieses Package ist gezielt von der API-Logik getrennt. Verschiedene Aufgaben werden von den Workers behandelt und durch Datenbank-Helpers in der Datenbank dokumentiert und gespeichert. Die Trennung des MLCores von der API ist gezielt, um die Verantwortlichkeiten der einzelnen Komponenten klar zu begrenzen und um die Skalierbarkeit sowie die Wiederverwendbarkeit der ML-Komponente für Batch-Training und Vorhersage zu realisieren. \cite{pedregosaScikitlearnMachineLearning2011}\cite{lundbergUnifiedApproachInterpreting2017}

Die Hauptdesignziele des Komponenten sind:
\begin{itemize}
    \item \textbf{Wiederverwendbarkeit (Reusability)}: Vordefinierte Preset-basierte Pipelines enthalten das Trainings- und Vorverarbeitungs-logik für verschiedene Klassifikations- und Regressions-modelle. Diese Pipelines werden sowohl für das Training als auch für die Vorhersage verwendet.
    \item \textbf{Erweiterbarkeit (Extensibility)}: Presets werden dynamisch aus mlcore/presets geladen. Das ermöglicht es, durch die Erstellung neuer Presets, neue Algorithmen als separate Modulen einfach einzufügen, ohne die Trainings- oder Vorhersagelogik anzupassen.
    \item \textbf{Testbarkeit (Testability)}: Profiling, Evaluation und Explainability sind als eigene Module erstellt, um eine einfache und gezielte Testbarkeit sicherzustellen.
\end{itemize}

% ########################################################

\subsection{Data Profiling und Vorverarbeitung}

\subsubsection{Data Profiling}

Das \code{mlcore/profile/profiler} berechnet statistische Daten und semantische Typen für die Spalten einer CSV-Datei. Für jede Spalte werden die folgenden Informationen berechnet:
\begin{itemize}
    \item der semantische Typ aus den Datentypen
        \begin{itemize}
            \item numeric
            \item categorical
            \item boolean
            \item datetime
            \item unknown
        \end{itemize}
    \item der Prozentanteil der fehlenden Werte, die Kardinalität und eine Zusammenfassung der Verteilung.
    \item ob alle Werte der Spalte gleich sind (konstante Spalte) oder ob die Spalte id-ähnlich ist (inkl. sequenzähnliche Integer-IDs) zur Exklusion.
    \item ob die Spalte sehr hohe Kardinalität hat, um eine Meldung an den Benutzer zu geben.
    \item eine vorgeschlagene Analyseart (Klassifikation oder Regression).
\end{itemize}

Das resultierende Profil enthält eine Datensatzzusammenfassung, die id-ähnliche Spalten (\glqq id\_candidates\grqq), einen Exklusionsvorschlag(\glqq exclude\_suggestions\grqq) und Metadaten pro Spalte. Später wird dieses Profil für die Vorverarbeitung und das semantische Typing während des Trainingsprozesses verwendet.

\subsubsection{Vorverarbeitung}

Das Training beginnt beim Laden der CSV-Datei einer Dataset-Version durch \code{data\_reader.get\_dataframe\_from\_csv}. Wenn es schon ein Profil für diese Dataset-Version gibt, liest der Trainer dieses ebenfalls ein, und wenn nicht, führt der Trainer die Funktion \code{profiler.suggest\_profile} aus, um eines zu berechnen. Dann wird die Vorverarbeitung durch \code{data\_reader.preprocess\_dataframe} ausgeführt. Die Vorverarbeitungsschritte sind die folgenden:

\begin{itemize}
    \item Der Benutzer kann durch die Feature-Strategie des ML Problems auswählen, welche Spalten inkludiert und welche exkludiert werden sollen. Das Default-Verhältnis der Feature-Stategie ist \glqq auto\grqq. Das bedeutet, dass alle Spalten als inkludiert und die Spalten aus \glqq exclude\_suggestions\grqq als exkludiert berücksichtigt werden. Dieses Verhältnis kann der Benutzer durch Anpassung der Feature-Strategie beeinflussen und nur die ausgewählten inkludierten und exkludierten Spalten berücksichtigen.
    \item Die Zeilen mit fehlenden oder Null-Werten in der Zielspalte werden entfernt.
    \item Die Daten werden in Merkmale X und Zielvariable y getrennt und zurückgegeben.
\end{itemize}

Am Ende werden die Spalten durch \code{data\_reader.get\_semantic\_types} mithilfe des Profils in kategoriale, numerische und boolische Listen unterschieden, die dann im ColumnTransformer in jedem Preset verwendet werden, um eine konsistente Behandlung für jeden Merkmalstyp sicherzustellen.

% ########################################################

\subsection{Training Pipeline Architektur}

Das Training wird durch \code{trainer.train} ausgeführt und folgt einer strukturierten Pipeline:
\begin{enumerate}
    \item Die Dataset-Version, Zielspalte und der Analysetyp werden aus der Datenbank geladen.
    \item Das Dataset-Version-Profil wird geladen oder aus der CSV-Datei neu erstellt.
    \item Die Daten werden vorverarbeitet und in X und y getrennt.
    \item Die semantischen Typen der Merkmale werden mithilfe des Profils bestimmt, da sie für die Konfiguration der Vorverarbeitungsschritte vor dem Modelltraining benötigt werden .
    \item Die Daten werden in Train/Test mit train\_test\_split\ getrennt (inkl. Statifikation für Klassifikation).
    \item Label Encoding wird auf die Zielspalte angewendet (nur für Klassifikation).
    \item Das Preset wird geladen und die Modell-Pipeline für den ausgewählten Algorithmus wird gebaut.
    \item Das Modell wird trainiert.
    \item Das Modell wird mit den Testdaten evaluiert.
    \item Die Cross-Validierungsmetriken werden berechnet (wenn gefordert).
    \item Eine Explainability-Zusammenfassung wird mithile des SHAP-Frameworks berechnet (wenn gefordert).
    \item Das Modell und seine Metadaten (inkl. Metriken, Cross-Validierung und Explainability-Zusammenfassung) werden gespeichert.
\end{enumerate}

\subsubsection{Preset Integration}

Die Presets werden dynamisch durch \code{preset\_loader.loader} geladen, der Module aus <algorithm>.py lädt. Jedes Preset bietet eine \code{build\_model()} Funktion, die die folgenden Objekte zurückgibt:
\begin{itemize}
    \item Eine scikit-learn-Pipeline, bestehend aus einem \code{ColumnTransformer} und einem \code{Estimator}.
    \item Ein Metadaten-Dictionary, das auf \code{metadata\_presets.py} basiert.
\end{itemize}
Dieser Prozess standardisiert die Vorverarbeitungs- und Trainingsprozesse für alle unterstützten Algorithmen und hält nur die estimator-spezifischen Parameter modular, um eine konsistente Struktur zu gewährleisten.

\subsubsection{Holdout Evaluation (Train-Test-Split)}

Nach dem Training des Modells werden Vorhersagen für die Testdaten erzeugt, um die Modellmetriken zu bestimmen. Für Klassifikationsaufgaben werden die Vorhersagen mithilfe des \code{LabelEncoder} wieder in die ursprünglichen Klassenlabels zurücktransformiert, und anschliessend werden die Metriken auf Basis der dekodierten Labels berechnet.

\subsubsection{Cross-Validation}

Wenn eine Cross-Validation gefordert wird (\code{evaluation\_strategy = “cv”}), wird zusätzlich durch mlcore/metrics/cv\_calculator eine Cross-Validation-Metrik berechnet:

\begin{itemize}
    \item Für Klassifikation wird die Macro-F1-Metrik mithilfe des Stratified K-Fold-Verfahrens berechnet.
    \item Für Regression wird die $R^2$-Metrik mithilfe des K-Fold-Verfahrens berechnet.
\end{itemize}
Falls das \glqq auto\grqq-Preset ausgewählt wurde, wird im Rahmen der Modellauswahl bereits eine Cross-Validation-Metrik über die verschiedenen Modelle hinweg berechnet, um das \glqq beste\grqq Modell zu bestimmen. In diesem Fall verwendet der Trainer diese intern-berechnete Cross-Validation-Zusammenfassung, anstatt die Cross-Validation erneut auszuführen.

% ########################################################

\subsection{Modell-Presets und Modellauswahl}

\subsubsection{Preset-basiertes Design}

Die Presets enthalten verschiedene Algorithmusoptionen und Vorverarbeitungslogik unter einer konsistenten Struktur, um zu gewährleisten, dass sie in standardisierter Weise angewendet werden können. Sie unterstützen verschiedene \code{training\_modes} (\glqq fast\grqq, \glqq balanced\grqq, \glqq accurate\grqq), die die Hyperparameter der Modelle anpassen, wie z.B. die Anzahl der Estimatoren oder die Regularisierungsparameter usw.

\subsubsection{Implementierte Modellfamilien}
\begin{itemize}
    \item \code{Linear / Logistic Modelle}
    \begin{itemize}
        \item Klassifikation: \code{LogisticRegression}
        \item Regression: \code{LinearRegression}, \code{Ridge}, \code{Ridge} in Kombination mit \code{PolynomialFeatures}
    \end{itemize}
    Begründung: Schnelle Baseline-Modelle mit hoher Interpretierbarkeit und robuster Leistung bei linear separierbaren Daten.
    \item \code{Entscheidungsbaum-basierte Modelle}
    \begin{itemize}
        \item Klassifikation: \code{RandomForestClassifier}, \code{ExtraTreesClassifier}
        \item Regression: \code{RandomForestRegressor}
    \end{itemize}
    Begründung: Nichtlineare Modellierung, robust gegenüber Merkmalskalierung sowie hohe Leistungsfähigkeit bei Datensätzen mit gemischten  Merkmaltypen.
    \item \code{Gradient-Boosting Modelle}
    \begin{itemize}
        \item Klassifikation: \code{HistGradientBoostingClassifier}
        \item Regression: \code{HistGradientBoostingRegressor}
    \end{itemize}
    Begründung: Hohe Leistungsfähigkeit bei tabularischen Daten bei gleichzeitig vertretbaren Trainingskosten. Diese Modelle erfordern dichte Eingabematrizen. Daher wird die Ausgabe des OneHotEncodings in diesen Presets explizit als \glqq dense\grqq konfiguriert.
    \item \code{XGBoost Modelle}\cite{chenXGBoostScalableTree2016}
    \begin{itemize}
        \item Klassifikation: \code{XGBClassifier}
        \item Regression: \code{XGBRegressor}
    \end{itemize}
    Begründung: Sehr hohe Leistungsfähigkeit bei tabularischen Daten durch starke Regularisierung und gradientbasiertes Tree-Boosting.
    \item \code{\glqq Auto\grqq-Preset}
    \begin{itemize}
        \item Klassifikation: \code{AutoClassifier}
        \item Regression: \code{AutoRegressor}
    \end{itemize}
    Begründung: Sie evaluieren mehrere Kandidatenmodelle aus einigen der zuvor genannten Algorithmen mithilfe von Cross-Validation-Metriken und wählen den Estimator mit der höchsten Leistung aus.
\end{itemize}

\subsubsection{Klassifikationspresets gegenüber Regressionspresets}

Klassifikationspresets verwenden probabilistische Ausgaben sowie die Macro-F1-Metrik, um Klassenungleichgewichte angemessen zu berücksichtigen.
Regressionspresets verwenden die $R^2$-Metrik und bieten optional zusätzliche \code{PolynomialFeatures}, um leichte Nonlinearitäten zu modellieren, während die Interpretierbarkeit weitgehend erhalten bleibt.

% ########################################################

\subsection{Evaluierung und Metriken}

\subsubsection{Klassifikationsmetriken}

Das \code{mlcore/metrics/metrics\_calculator} berechnet für Klassifikation:

\begin{itemize}
    \item Genauigkeit (\glqq Accuracy\grqq)
    \item Präzision (\glqq Precision\grqq) (macro)
    \item Sensitivität (\glqq Recall\grqq) (macro)
    \item F1 (macro)
\end{itemize}

Macro-averaging wird verwendet, um die Übergewichtung von Mehrheitsklassen zu vermeiden und eine gleichmässige Bewertung aller Klassen sicherzustellen.

\subsubsection{Regressionsmetriken}
Das \code{mlcore/metrics/metrics\_calculator} berechnet für Regression:

\begin{itemize}
    \item Mean Absolute Error (\glqq MAE\grqq)
    \item Mean Squared Error (\glqq MSE\grqq)
    \item Root Mean Squared Error (\glqq RMSE\grqq)
    \item $R^2$
    \item Mean Absolute Percentage Error (\glqq MAPE\grqq)
\end{itemize}
Diese Metriken bieten eine komplementäre Sicht auf die Fehlersensitivität und das relative Fehlerverhältnis.

\subsubsection{Einschränkung der Metriken}
\begin{itemize}
    \item Die Macro-F1-Metrik gewichtet jede Klasse gleich und kann daher die Leistung auf dominanten Klassen unterschätzen.
    \item $R^2$ kann bei stark nichtlinearen Zusammenhängen oder schief verteilten Zielvariablen die Modellgüte falsch darstellen.
    \item MAPE wird instabil, wenn die tatsächlichen Zielwerte nahe Null liegen.
\end{itemize}

% ########################################################

\subsection{Explainability}
Die Explainability wird in \code{mlcore/explain} implementiert und basiert vollständig auf dem SHAP-Framework. Der Trainer erstellt einen SHAP-Explainer im transformierten Merkmalsraum nach Anwendung des ColumnTransformer, wobei die Merkmalsnamen aus dem trainierten ColumnTransformer übernommen werden.

\subsubsection{SHAP Workflow}

\begin{itemize}
    \item Ein Referenzdatensatz wird als Stichprobe aus den Trainingsdaten erzeugt.
    \item Ein Erklärungsdatensatz wird als Stichprobe aus den Testdaten erzeugt.
    \item Der shap.Explainer wird entweder für \code{predict\_proba} (für Klassifikation) oder für \code{predict} (für Regression) konfiguriert.
    \item In \code{summary\_calculator.py} wird eine statistische Zusammenfassung erstellt, die Folgendes enthält:
    \begin{itemize}
        \item Globale Mean Absolute SHAP-Werte pro Merkmal zur Identifikation der Top-$k$-wichtigsten Merkmale (Standardwert $k=30$).
        \item Aggregation auf Elternebene zur Gruppierung von durch OneHotEncoding erzeugten Merkmalen.
        \item Quantil-Zusammenfassungen von SHAP-Werten und Merkmalswerten (dieses Feature wurde verworfen und nicht weiterverwendet).
    \end{itemize}

\end{itemize}

\subsubsection{Merkmalsnamen und Gruppierung}
In \code{get\_feature\_names} werden die Merkmalsnamen aus der Vorverarbeitungspipeline rekonstruiert:
\begin{itemize}
    \item One-Hot-encodierte kategoriale Merkmale werden als \code{\glqq col=value\grqq}-Einträge dargestellt.
    \item Numerische und boolesche Merkmale werden mit ihren ursprünglichen Namen dargestellt.
    \item Eine parallele Liste feature\_parents erhält die Gruppierung nach der ursprünglichen Spalte.
\end{itemize}

% ########################################################

\subsection{Vorhersage Pipeline Architketur}

Die Vorhersage wird durch \code{predictor.predict} ausgeführt:

\begin{enumerate}
    \item Die Eingabedaten werden aus einer CSV-Datei oder einem In-Memory-DataFrame geladen.
    \item Das ausgewählte Modell wird geladen:
    \begin{itemize}
        \item Wenn \code{model\_id=\glqq production\grqq}, wird das aktuelle Produktionsmodell des gegebenen ML-Problems geladen. Um diese Option zu verwenden, muss bereits ein Modell für dieses ML-Problem mit dem Status “Production” gesetzt sein.
        \item Andernfalls wird das Modell mit der gegebenen ID geladen.
    \end{itemize}
    \item Die Modellmetadaten und das gespeicherte Zielschema werden für die Datenvalidierung geladen.
    \item Die Zielspalte wird aus den Eingabedaten verworfen, wenn sie vorhanden ist.
    \item Das Schema der Eingabedaten wird validiert:
    \begin{itemize}
        \item Spalten mit leerem oder dupliziertem Namen werden verworfen.
        \item Alle Merkmale aus dem \code{feature\_order} des Schemas sind erforderlich.
        \item Zusätzliche Spalten werden geloggt und anschliessend ignoriert, indem eine Neuordnung gemäss dem gespeicherten \code{feature\_order} erfolgt.
    \end{itemize}
    \item Die Vorhersage erfolgt mithilfe der geladenen Pipeline.
    \item Für Klassifikation werden die Vorhersagen durch \code{label\_classes} zurücktransformiert, sofern diese in den Metadaten gespeichert sind.
    \item Die Ergebnisse werden in einer Vorhersagezusammenfassung mit Eingabe- und Ausgabedaten gespeichert.
\end{enumerate}

Die Vorhersageausgabe enthält die verwendeten Eingabedaten, die vorhergesagten Labels (für Klassifikation) sowie eine Kopie der Modellmetadaten zur Nachvollziehbarkeit und Kontrolle.

\subsection{Einschränkungen und mögliche Erweiterungen}

\subsubsection{Skalierbarkeitseinschränkungen}
\begin{itemize}
    \item Training und Explainability werden im Hauptspeicher (in-memory) und auf der CPU ausgeführt und skalieren daher nur eingeschränkt für sehr grosse Datensätze. 
    \item SHAP-Berechnungen können auch bei Verwendung von Stichproben rechenintensiv sein, insbesondere bei grossen Merkmalsräumen.
\end{itemize}

\subsubsection{ifecycle Verwaltung}
\begin{itemize}
    \item Modell- und Vorhersageversionen werden über Metadaten und Datenbankeinträge verwaltet, jedoch existiert keine automatisierte Retention- oder Retirement-Strategie zur Bereinigung veralteter Modelle oder Vorhersagen, was langfristig zu unnötigem Speicherverbrauch führen kann.
    \item Es ist keine automatische Überwachung der Modelle und ihrer Leistung in der Produktionsumgebung implementiert. Veränderungen in den Vorhersageeingabedaten oder Leistungsabfälle der Modelle könnten daher unentdeckt bleiben.
\end{itemize}

\subsubsection{Erweiterungen}
\begin{itemize}
    \item Einführung eines kontinuierlichen Modell-Monitorings zur Überwachung von Datenverteilungen, Vorhersageverhalten und Leistungsmetriken in der Produktionsumgebung.
    \item Erweiterung der Presets und “Auto”-Presets um systematische Hyperparameteroptimierung, z.B. durch GridSearchCV, RandomizedSearchCV oder Optuna.
    \item Einführung von speicher- und rechenoptimierten Trainingsverfahren, z.B. durch Batch-Verarbeitung oder GPU-Unterstützung zur Verbesserung der Skalierbarkeit.
\end{itemize}


\section{User Experience}
\label{sec:ux}

\subsection{FastAPI}
\label{subsec:ux:fastapi}
\lipsum

\subsection{WebUI}
\label{subsec:ux:webui}
\lipsum
\section{Perspektive}
\label{sec:perspektive}
\lipsum

\subsection{Skalierbarkeit CSV-Upload}
\label{subsec:csv}
\lipsum
\section{Fazit}
\label{sec:fazit}
\code{PAaaS} bietet eine verständliche, modulare und skalierbare Lösung für eine API und Weboberfläche zur Nutzung von  prädiktiven Analyseverfahren. Dabei konnte ein Großteil der gesetzten Ziele erfolgreich umgesetzt werden. Im Rahmen einer Weiterentwicklung des Projektes gibt es interessante Erweiterungs- und Integrationsmöglichkeiten. Die bestehenden Funktionalitäten von \code{PAaaS} bieten eine klare Vision, wie prädiktive Analysen nutzerfreundlich durchgeführt werden können.

%----------------------------------------------------------------------------------------
%	 REFERENCES
%----------------------------------------------------------------------------------------

% \clearpage
% \scriptsize
\section*{Anmerkung zur Nutzung von KI-Tools}
In diesem Bericht wurden KI-Tools (\glqq ChatGPT\grqq) zur Fehlerkorrektur und sprachlichen Ausgestaltung verwendet. \glqq GPT-5.2 Thinking\grqq\space wurde zur Unterstützung bei Formulierungen, Struktur und Grammatik eingesetzt. Der Projekt-Code verwendet in Teilen Code-Snippets, bei denen eine KI-Verwendung nicht ausgeschlossen werden kann (zum Beispiel durch in Suchmaschine eingebundene Tools wie \glqq Google Search\grqq, \glqq Microsoft Copilot\grqq, \glqq duck.ai\grqq).
\listoffigures
% \listoftables
% \AtNextBibliography{\scriptsize}
\printbibliography % Output the bibliography

%----------------------------------------------------------------------------------------

\end{document}
