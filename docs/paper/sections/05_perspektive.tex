\section{Perspektive}
\label{sec:perspektive}
\lipsum
 
\subsection{MLCore}
In MLCore sind besonders folgende Punkte als mögliche Erweiterungen und Ausbaumöglichkeiten zu betrachten:
\begin{itemize}
    \item Einführung eines kontinuierlichen Modell-Monitorings zur Überwachung von Datenverteilungen, Vorhersageverhalten und Leistungsmetriken in der Produktionsumgebung.
    \item Erweiterung der Presets und “Auto”-Presets um systematische Hyperparameteroptimierung, z.B. durch GridSearchCV, RandomizedSearchCV oder Optuna.
    \item Einführung von speicher- und rechenoptimierten Trainingsverfahren, z.B. durch Batch-Verarbeitung oder GPU-Unterstützung zur Verbesserung der Skalierbarkeit.
\end{itemize}

\subsection{Skalierbarkeit CSV-Upload}
\label{subsec:csv}
\lipsum

\section{Anmerkung zur Nutzung von KI-Tools}
In diesem Bericht wurden KI-Tools (\glqq ChatGPT\grqq) zur Fehlerkorrektur und sprachlichen Ausgestaltung verwendet. Der Projekt-Code verwendet in Teilen Code-Snippets, bei denen eine KI-Verwendung nicht ausgeschlossen werden kann (zum Beispiel durch in Suchmaschine eingebundene Tools wie \glqq Google Search\grqq, \glqq Microsoft Copilot\grqq, \glqq duck.ai\grqq)