\section{Perspektive}
\label{sec:perspektive}
\lipsum
 
\subsection{MLCore}
In MLCore sind besonders folgende Punkte als mögliche Erweiterungen und Ausbaumöglichkeiten zu betrachten:
\begin{itemize}
    \item Einführung eines kontinuierlichen Modell-Monitorings zur Überwachung von Datenverteilungen, Vorhersageverhalten und Leistungsmetriken in der Produktionsumgebung.
    \item Erweiterung der Presets und “Auto”-Presets um systematische Hyperparameteroptimierung, z.B. durch GridSearchCV, RandomizedSearchCV oder Optuna.
    \item Einführung von speicher- und rechenoptimierten Trainingsverfahren, z.B. durch Batch-Verarbeitung oder GPU-Unterstützung zur Verbesserung der Skalierbarkeit.
\end{itemize}

\subsection{Frontend}
Auch im Bereich des Frontends / der WebUI sind Verbesserungsmöglichkeiten vorhanden:
\begin{itemize}
    \item Jobs-Seite: Die Jobs-Seite ist derzeit lediglich ein Platzhalter und bietet keine detaillierten Echtzeit-Ansichten der Hintergrundprozessen.
    \item Auswahl existierender CSV-Dateien: Die Formulare zur Erstellung von Datensatz-Versionen sowie zur Vorhersage enthalten einen Modus zur Verwendung einer existierenden CSV-Datei. Dieser ist aktuell deaktiviert, da noch keine vollständige Unterstützung für gespeicherte CSV-Dateien implementiert ist.
    \item Explainability-Diagramme: Wie bereits im MLCore-Explainability-Kapitel erwähnt, waren zusätzliche Visualisierungen (z.B. Quantil-Zusammenfassungen) vorgesehen. Diese wurden aufgrund zeitlicher Einschränkungen nicht umgesetzt, und sind eine mögliche Erweiterung zur vertieften Analyse.
\end{itemize}

\subsection{Skalierbarkeit CSV-Upload}
\label{subsec:csv}
\lipsum
