\section{Struktur}
\label{sec:struktur}
\ja
Bei dem Design des Projektes wurde ein großer Fokus auf Modularisierung und Skalierbarkeit gelegt. Ebenso wurden primär weit verbreitete Tools eingesetzt, um die Weiterentwicklung und Wartung des Projektes zu vereinfachen. Folglich sind die einzelnen Komponenten über bestimmte Schnittstellen miteinander verbunden. Innerhalb dieser ist eine Modifizierung oder ein Austausch der Komponenten unabhängig der anderen Komponenten umzusetzen. Dabei wurde Docker als Containerisierungslösung eingesetzt, da Docker eine sehr weit verbreitete Lösung darstellt \cite{sollfrankEvaluatingDockerLightweight2021}.

\subsection{Komponenten}
\label{subsec:komponenten}
\ja
Die einzelnen Komponenten des Projekts werden als containerisierte Microservices ausgerollt. Dabei werden die folgenden Komponenten verwendet:

\begin{enumerate}
    \item Frontend Node: eine React-basierte Weboberfläche zur vereinfachten Nutzung durch den Anwender
    \item API Node: eine FastAPI-basierte API, über welche sämtliche Funktionen des Projektes angestoßen werden können
    \item Worker Node: ein Python-Container, in dem die MLCore-bezogenen Tasks ausgeführt werden
    \item MySQL: eine SQL-basierte Datenbank
    \item Redis: Messaging Host für Zuweisung von Celery Tasks von / an verschiedene API / Worker Nodes
    \item Flower: eine Weboberfläche zum Redis-Monitoring
\end{enumerate}

Besonders sind dabei die Frontend, API und Worker Nodes zu betrachten. Die verbleibenden Komponenten sind unmodifizierte 3rd-party Softwares.

% TODO:
\todo{Aufbau von frontend, API, Worker. Warum wurden diese Tools gewählt?}

MySQL wurde aufgrund der weiten Verbreitung des Tools verwendet. So war MySQL 2023 die zweit-meistverwendeste Datenbank nach Oracle \cite{DBEnginesRanking}. Oracle wurde allerdings aufgrund der Lizenzbedingungen für dieses Projekt nicht in Betracht gezogen. Zudem wurde speziell eine SQL-basierte Datenbank verwendet, um die Vorteile einer effizienten Speicherung für strukturierte Metadaten zu nutzen. Dabei ist zu beachten, dass in der aktuellen Version Nutzer-Uploads nur eingeschränkt in der Datenbank gespeichert werden (siehe Sektion \ref{sec:perspektive}). Für eine direkte Speicherung z.B. von .csv-Uploads wären auch dokumentbasierte Datenbanken wie beispielsweise MongoDB in Betracht zu ziehen gewesen \cite{DocumentStoresDBEngines}.

Redis und Celery sind verbreitete Tools für Messaging. Diese Tools wurden aufgrund bestehender Erfahrungen im Team sowie der leichten Implementierung in Python gewählt. Flower wurde folglich als Monitoring-Tool für Redis gewählt.

\begin{figure*}
    \centering
    \includegraphics[width=\textwidth]{../structure_release.png}
    \caption{Containerisierte Projekt-Struktur}
    \label{fig:structure}
\end{figure*}

\subsection{Containerisierung}
\label{subsec:container}
\ja
\lipsum

\subsection{Datenbank}
\label{subsec:db}
\hg
\lipsum

\section{Workflow}
\ja
\label{sec:workflow}
% TODO: pytest, CI
\lipsum