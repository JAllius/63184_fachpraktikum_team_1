Im Rahmen des Fachpraktikums \glqq Softwareentwicklung mit Methoden der Künstlichen Intelligenz\grqq\space wurde das Projekt \glqq Predictive Analytics-as-a-Service\grqq\space (PAaaS) erstellt. Dabei handelt es sich um eine skalierbare Web-Anwendung, um die Anwendung verschiedener KI-Algorithmen zur Analyse diverser Datensätze zu vereinfachen. In einer containerisierten Microservice-Architektur wird unter anderem FastAPI, React und Celery / Redis verwendet, um das \glqq MLCore\grqq\space genannte Backend anzusteuern. MLCore bietet dabei viele Möglichkeiten für predictive analysis und reduction analysis. Neben erweiterten Analyseoptionen liegen mögliche Verbesserungen im Bereich der Einstiegshilfe, der internen Datenspeicherung und der allgemeinen Nutzungsleichtigkeit vor.