\section{Übersicht}
\label{sec:uebersicht}
Zunächst wird kurz die Idee von \code{PAaaS} und der Aufbau betrachtet. Dabei wird auf die einzelnen Komponenten und deren Zusammenspiel in Form von Containerisierung eingegangen. Auch wird der Workflow während der Projekt-Entwicklung dargestellt. Anschließend wird detailliert auf den Aufbau der Datenbank und die Strukturierung der Daten eingegangen. Danach folgt eine detaillierte Darstellung von MLCore, also dem Modell-Training und der eigentlichen \glqq Predictive Analysis\grqq. Abschließend wird auf einige Erweiterungen und Ausbaumöglichkeiten eingegangen. Diese stellen interessante Optionen für eine Weiterentwicklung dar, welche jedoch im Zeitrahmen des Projektes nicht mehr umgesetzt werden konnten. 

% \subsection{Anmerkung der Autoren}
% \label{subsec:anmerkung}
% Dieser Bericht sowie der \code{PAaaS}-Code wurde im Rahmen eines Fachpraktikums an der Fernuniversität Hagen umgesetzt. Dabei wurden innerhalb des Projektes die Aufgaben aufgeteilt.

% Die Komponenten Frontend sowie MLCore wurden durch Nikolaos Psycharis implementiert. Das Projektdesign, die Datenbank sowie Anbindung dieser an FastAPI und MLCore wurden durch Hamza Gueni umgesetzt. Die Containerisierung/Skalierung und Endpoints wurden durch Joshua Allius umgesetzt.

\subsection{Idee}
\label{subsec:idee}
\textcolor{red}{\lipsum[4]}